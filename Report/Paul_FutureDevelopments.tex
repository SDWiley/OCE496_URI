\section {Finite Element Model}

\subsection{Model Improvements}

The Abaqus model produced for this project can be further developed to produce more accurate modal responces to static and dynamic loading. Any improvement to the model will replicate the bridge more closely. The longterm effect of fatigue is immeasurable and will create an unavoidable error. After measuring the current natural frequency of a bridge an origional sitffness matrix can be estimated from the bridge a sfittness matrix can be estimated and inputed to Abaqus, however, evaluation the effects of fatigue on structural integrity is difficult.    
\indent The current model is considered to be one cohesive piece with a uniform stiffness rather than different indivudual sections that are bolted or welded together. In a more precise model, as the 19,0000 element model produced for evaluating the Tsing Ma Bridge in Hong Kong, each piece of the bridge was modeled to reflect the actual material. The steel cables were molded as thousands of steel wires and the road deck was modeled as pavement \cite{Chan}. As steel is primarly used for the entire structure of the Claiborn Bridge the error aquired is relatively small. 
\indent Suspension bridges will deform plastically everywhere accept for the welded regions which should be modled as ridgid pieces. These joints are critical locations in the system as fatigue cracks will develop first. If the welds that are under the most stress can be identified they can be reinforced or monitored more closely \cite{Chan}. Bridge secions that connect at bolts rather than welds have friction at the interface of the two sections. This friction is imporatant when evaluating failure.

\subsection{Dynamic Loading} 

\indent In this evaluation, only static loading was evaluated as dynamic loading is outside the scope of this project. The model of the Tsing Ma Bridge was evaluated for dynamic loading as two mediums of traffic utalize this bridge. The trains and trucks that travel over the bridge require entirely seperste analysis as they present the bridge with entirely different loading signatures.\\
\indent The loading of a train, individual truck in both lanes, and groups of trucks with varying orientation were simulated passing over the bridge. As a single truck will not affect the net dynamic response only the local response must be examined. To minimize computation time only relevant elements were applied a tapering load for less than 2 s and then eliminated from the simulation. Groups of trucks were separated by a 3 s lag time. The variability of traffic was examined and the co-existence of the two types of loading were quantified. A stress cycle was produced for each traffic variation and compiled to identify which element would experience the most intense local stresses. The critical locations of local stress under truck loading are the outmost part of the upper chord and the bottom cross-frame between the rail tracks. Under train loading the critical locations in the deck unit are at the outmost parts of the upper chord and show more stress than the highest local stress under truck loading. Because of this it was determined that an outbound train will produce the most stress in the upper chords along the bridge longitudal direction. This location was chosen to determine fatigue critical locations in the whole bridge \cite{Chan}. 
