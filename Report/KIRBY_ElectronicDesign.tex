\subsection{Introduction}
\indent The unique requirements for the sensor developed for deployment on the Claiborn Pell Newport Bridge set the sensor apart from off-the-shelf sensors readily available for purchase.
The sensor package needed to record high precision, high resolution accelerometer and strain gauge data continuously for an extended period of time. 
The longevity of the package was dependent upon the battery capacity and data storage capacity. 
This could have been solved by utilizing a large bank of batteries and multiple hard disk drives; however it was determined that this was a not a feasible option. 
Instead the sensor package would scavenge energy to recharge batteries and transmit data to a base station wirelessly. 
The addition of these two requirements greatly increased the complexity of the sensor package design. \\

\indent 

\subsection{Circuitry}

\subsubsection{Voltage Regulation}
The voltage input for most systems on the sensor board are a range of voltages between 3.3V-5V.
This posed a basic issue due to the output voltage of the 12V battery. 
The solution was to use two LM317 linear voltage regulators.
The LM317 technical specifications are displayed in Table \ref{tab:LM317} 

\begin{table}[h]
\centering
\begin{tabular}{|l|c|}
\hline
\textbf{Parameter} & \textbf{Value}\\
\hline
Input Voltage Differential ($V_{in}-V_{out}$)& 3V $\le V_{in}-V_{out} \le$ 40V\\
Output Voltage ($V_{out}$) & 1.2V $\le V_{out} \le$ 37V\\
Output Current ($I_{out}$) & 1.5A\\
Max Power Dissipation ($P_{D}$) & 20W\\
Package Type				   & TO-220\\
\hline
\end{tabular}
\caption{LM317 Adjustable Linear Regulator Specifications}
\label{tab:LM317}
\end{table}

The output voltage can be set using Equation \ref{eqn:LM317} where $R_1$ and $I_{adj}$ are typically $240\Omega$ and $\le 100\mu A$ respectfully. It should be noted the $V$ is not the input voltage, but a unit placeholder and 
\begin{equation}
V_{out} = 1.25V(1+\frac{R_2}{R_1}) + I_{adj}R_2
\label{eqn:LM317}
\end{equation}

\subsection{Printed Circuit Board}