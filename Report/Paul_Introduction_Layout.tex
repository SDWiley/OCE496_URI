%\chapter*{Layout} 

\indent This report will discuss the planning process of designing a sensor package intended to evaluate vibration on the Claiborne Pell Bridge. As this is a
two-semester project the preliminary assembly of the package was simplified to evaluate vibrations of a 6.8 meter angle beam in phase one. A finite element
model was produced of this angle beam to provide information on the modes of vibration and natural frequency. In the second semester the package was
further developed but not completed. Data was collected from the top of the towers by the incomplete package. A FEM was produced of the Claiborne Bridge
and used to evaluate approximate natural frequencies of the bridge and to indicate where the package should be installed for best results. \\

\indent Chapter \ref{ch:FEM} will explain the process the finite element model that was produced for both the angle beam and the Claiborne Bridge. The specific
parameters  used to describe the material are described in detail. To prove that the finite element model was reflecting accurate natural frequencies and
modes of vibration, the model for the angle beam was verified with an analytical solution. \\
\indent Choosing the appropriate, cooperating instrumentation is integral to producing a sensor package that will accurately monitor vibration frequencies.
Chapter \ref{ch:Instrumentation} of this report will present the instrumentation chosen and why. \\
\indent Three separate data sets were recorded; that of the angle beam, the bridge, and the battery discharge curve. The collection process and processing
details are discussed in Chapter \ref{ch:DataCollection}. \\ 
\indent The verifications and comparisons of the various data sets collected are discussed in Chapter \ref{ch:DataAnalysis} along with other discussion about the data
%This next paragraph is quite weird Elizabeth :P 
\indent As the package was not finished to completion, the future developments that are required to make this package whole are indicated in Chapter \ref{ch:FutureDevelopment}.
There were four systems that were not implemented; the GPS time synchronization, wireless communication capabilities, integration of the strain gauge, and
power independence. Chapter \ref{ch:FutureDevelopment} describes the measures that need to be taken to complete the sensor package.\\