\subsubsection{Integration of Strain Gauge}

\indent Figure \ref{fig:OmegaSoldered} is one of the 3-element rosette strain gauges with pre-soldered ribbon leads. Purchased from omega engineering with
model number SGD-6/120RYT23. These strain gauges were necessary after realizing the difficulty of soldering leads to the original set of strain gauges. The
pre-soldered ribbon leads on the second set of strain gauges also proved to be unsuccessful during experimentation. This was because the leads would not
stay secured to the attaching clips while tests were being run. Even with the persistent attempts to get the strain gauges connected and working
correctly, the data received was still very inaccurate. One possible contributing factor to this may have be the lack of precision when applying the
strain gauge to the exact location on the beam. However, one definite factor that contributed to the inaccurate data from the strain gauges was the
type of strain gauge that was used. A strain gauge with a different gauge factor and a higher resistance would have been more favorable. The higher
the resistance of a strain gauge, the higher the sensitivity. The original sets of strain gauges had a resistance of 120 $\Omega$, but to precisely
measure strain on a beam the resistance must be much higher, 350 $\Omega$ or more. The costs for a pack of 6 similar strain gauges with a resistance of
350 $\Omega$ from omega engineering is one hundred dollars. Another factor that halted the efforts to apply the strain gauge was that they required
another ADC output. It is possible to make more outputs, however this also demands that the time synchronization is even more accurate.
Nevertheless, higher resistant strain gauges would be better for sensor packages for future developments. \\

\begin{figure}[h]
\centering
\includegraphics[width=0.3\textwidth]{Strain_Gauge_with_Leads.png}
\caption{Omega 3-Element Rosette with pre-soldered leads.}
\label{fig:OmegaSoldered}
\end{figure}