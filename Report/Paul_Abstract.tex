Structural Health monitoring is the process of evaluating and assessing fatigue of existing infrastructure by continuously monitoring a long term change in
natural frequency vibration.
Aging traffic infrastructure is a growing problem across the world as material and construction cost has risen considerably in
recent decades.
As replacement becomes more difficult reliable lifetime extension is possible through structural health monitoring.
At the completion of this project a sensor package was designed and partially assembled, a simple finite element model (FEM) of the main span was produces, high precision
GPS units were installed and collected data for two weeks, and preliminary data was collected with the package accelerometers on top of the towers of
the bridge.
With development and research damage location, type, and severity may be indicated by interpreting the vibration changes. 


